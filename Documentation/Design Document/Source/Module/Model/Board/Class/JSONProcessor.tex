\paragraph{Class JSONProcessor}\mbox{}
\begin{tabularx}{\textwidth}{|c||l|l|l|X|}
    \hline
    \cellcolor{lightgray}Class Name & \multicolumn{4}{X|}{JSONProcessor}\\
    \hline
    \cellcolor{lightgray}Inherits From & \multicolumn{4}{X|}{None}\\
    \hline
    \cellcolor{lightgray}Description & \multicolumn{4}{p{12cm}|}{Processes a .json file and return a JSON object. Using Simple JSON library}\\
    \hline\hline
    
    \cellcolor{lightgray}Methods & \cellcolor{lightgray}Visibility & \multicolumn{2}{l|}{\cellcolor{lightgray}Method Name} & \cellcolor{lightgray}Description\\\cline{2-5}
    \hline
    \cellcolor{lightgray} & Public & \multicolumn{2}{l|}{ProcessCurrentJSON()} & Processes the current .json file defined in Constants.\\
    \hline
    \cellcolor{lightgray} & Public & \multicolumn{2}{l|}{ProcessCustomJSON(Path path)} & Process any kind of .json file\\
    \hline
    \cellcolor{lightgray} & Public & \multicolumn{2}{l|}{ProcessJSON(Path path)} & Process .json object. If the specified file does not exist, the program will terminate and display the error\\
    \hline
\end{tabularx}